\chapter{Conclusion}\label{sec:conclusion}
In this paper, we presented CLMR, a self-supervised contrastive learning framework that learns useful and compact representations of raw waveforms of musical audio.
The framework requires no preprocessing of the input and is trained without ground truth, which enables simple and straightforward pre-training on datasets of unprecedented scale.
We tested the representations in the music classification task on the MagnaTagATune and Million Song Dataset, achieving state-of-the-art and competitive performance respectively, exceeding previous supervised benchmarks, and demonstrated the transferability of representations learned from pre-training on different datasets.
The simplicity of training the model without a direct supervised signal and without preprocessing inputs, together with encouraging results obtained with a single linear layer optimised for a challenging downstream task, are exciting developments towards unsupervised learning on raw musical signals in the time-domain.

\section*{Broader Impact}
This thesis describes a self-supervised learning framework for audio in the time-domain.
It can be used to learn useful representations without the need for human-annotated labels.
While this research has achieved results that will advance the field if Music Information Retrieval into a new learning paradigm, we would like to put forward the following recommendations:

\begin{itemize}
    \item Care must be taken when compiling a dataset of music for pre-training. The biases that are learned inherently in the CLMR model are not investigated in this research. These biases could lead to a potentially skewed prediction performance in favor of a select genre or type of music.
    \item The newly introduced method in this thesis, CLMR, requires no human-annotated labels to learn useful representations of music. While normally compensation is provided for human annotation, and, with the absence of this requirement, we recommend instead to compensate artists directly when training this algorithm on their music and when developing a commercial application.
\end{itemize}